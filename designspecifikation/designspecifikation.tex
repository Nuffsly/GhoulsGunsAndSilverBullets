\documentclass{TDP005mall}
\usepackage{graphicx}
\usepackage{float}


\newcommand{\version}{Version 1.0}
\author{Martin Kuiper, \url{marku849@student.liu.se}\\
  Jim Teräväinen, \url{jimte145@student.liu.se}}
\title{Designspecifikation}
\date{\today}
\rhead{Martin Kuiper\\
Jim Teräväinen}



\begin{document}
\projectpage
\section{Revisionshistorik}
\begin{table}[!h]
\begin{tabularx}{\linewidth}{|l|X|l|}
\hline
Ver. & Revisionsbeskrivning & Datum \\\hline
1.0 & Första utkast för designspecifikationen. & 2020-11-27 \\\hline
\end{tabularx}
\end{table}

\section{Klassbeskrivningar}
\subsection{Player}
Player ska kunna uppdatera sin position beroende på vilka knappar som är intryckta samt vilket state Player har.
Player har en överlagrad render()-funktion, då den behöver kunna uppdatera hur den renderas beroende på vad som händer Player-objektet.
Funktionen tick() kallar alla funktioner som ska köra när spelet uppdateras. 

\subsubsection{Datamedlemmar}
\begin{itemize}
\item state - påverkar hur objektet beter sig när tick() kallas.
\item weapon - ett objekt av klassen Weapon ägs av Player.
\item immune - hur länge Player ska vara osårbar.
\item health - ärvd från Character, minskar när Player tar skada.
\item damage - ärvd från Character, påverkar Players vapens skada.
\item speed - ärvd från Character, påverkar hastigheten av objektets förflyttning.
\item shape - ärvd från Textured Object, påverkar hur objektets textur ritas ut.
\item center - ärvd från Game\_Object, bestämmer objektets position på skärmen.
\item hitbox - ärvd från Game\_Object, bestämmer storleken på objektets kollisionsområde.
\end{itemize}

\subsubsection{tick(Time)}
För att flytta Player kallas move\_player().

För att flytta och skjuta vapnet kallas move\_weapon().

Player frågar World vilka objekt den kolliderar med. Kolliderar Player med ett Enemy-objekt kallas take\_damage().

Datamedlemmen 'immune' minskas med Time om den inte är 0.

\subsubsection{move\_player()}
Om någon av tangenterna som ska flytta spelarkaraktären till höger eller vänster är intryckta ska Players position uppdateras i den riktningen, så länge Player inte befinner sig vid banans kant.

Normalt har spelaren state 'standing'. 
Har spelaren state 'falling' flyttas spelaren mot botten av skärmen. 
Har spelaren state 'jumping' flyttas spelaren mot toppen av skärmen.

När spelaren trycker på hopp-knappen ändras state till 'jumping' om det inte redan är 'jumping' eller 'falling'.
När spelaren har haft state 'jumping' en viss tid ändras state till 'falling'.
Om den understa delen av Player-objektet kolliderar med den översta delen av en plattform sätts 'falling' till 'standing'.

\subsubsection{move\_weapon()}
Players position bestämmer Weapon-objektets position. 
Input från tangentbord eller muspekare bestämmer vapnets rotation.
Vapnet har en datamedlem 'cooldown' som påverkar hur ofta det kan avfyras. 
Är 'Avfyra vapen'-knappen intryckt och vapnets 'cooldown' 0 skapas en projektil.
Projektilen rör sig i riktningen av vapnets rotation och har en datamedlem 'damage' som sätts till Players 'damage'.

\subsubsection{take\_damage(int)}
Players 'health' subtraheras med skadan som skickas som parameter till funktionen. 

Datamedlemmen 'immune' sätts enligt hur länge Player ska vara odödlig efter att ha tagit skada.


\subsection{Hud}
HUD står för Heads-Up Display och har som syfte att visa relevant information för spelaren. 
Hud-klassen ska visa Player-objektets nuvarande hälsa (health) samt spelarens insamlade pengar.

\subsubsection{Datamedlemmar}
\begin{itemize}
\item player\_ptr - pekare till player för att kunna hämta dess hälsa.
\item shape - ärvd från Textured Object, påverkar hur objektets textur ritas ut.
\item center - ärvd från Game\_Object, bestämmer objektets position på skärmen.
\item hitbox - ärvd från Game\_Object, bestämmer storleken på objektets kollisionsområde.
\end{itemize}

\subsubsection{tick(Time)}
Player-objektets hälsa hämtas och en grafik med den informationen ritas ut i kanten av skärmen ovanpå alla andra objekt.

Mängden insamlade pengar hämtas från Game\_States datamedlem player\_info och renderar detta på liknande sätt. 

\end{document}
