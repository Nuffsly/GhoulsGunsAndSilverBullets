\documentclass{mall}

\newcommand{\version}{Version 1.0}
\author{Klas Arvidsson, \url{klas.arvidsson@liu.se}\\
  Pontus Haglund, \url{pontus.haglund@liu.se}\\
  Emma Enocksson Svensson, \url{emma.enocksson@liu.se}\\
  Filip Strömbäck, \url{filip.stromback@liu.se}}
\title{Gruppkontrakt}
\date{2020-10-01}
\rhead{}


\begin{document}
\projectpage

% När ni har fyllt i dokumentet kan denna rubrik tas bort helt.

\section*{Lära känna varandra}
\label{get-to-know}

Detta dokument fylls lämpligtvis i under ett möte där alla i gruppen är närvarande. Inför mötet
förbereder var och en av deltagarna det som beskrivs nedan och i avsnitt \ref{prereq}. Sedan följs
instruktionerna i varje nytt avsnitt.

Börja mötet med att låta alla i gruppen presentera sig själva. Förbered dig genom att fundera på
följande punkter och ta upp det som är relevant:

\begin{itemize}
\item Var har jag för kunskaper sedan tidigare som kan vara relevanta i projektet?
\item Vad har jag för ambitionsnivåer i projektet? Siktar jag på högt betyg och bra resultat, eller
  är jag mer intresserad av att få godkänt och inte så mycket mer?
\item Vad vill jag lära mig i projektet?
\item Hur lär jag mig saker bäst?
\item Hur känner jag angående planering? Är det viktigt att planera projektet i detalj, eller är det
  okej att ta saker som de kommer?
\item Vad har jag för fritidsintressen? Får projektet inkräkta på dem? Vill jag umgås med projektmedlemmar på fritiden?
\end{itemize}


\section{Förutsättningar}
\label{prereq}

% När ni har fyllt i dokumentet kan instruktionerna nedan tas bort.

Denna bit görs lämpligtvis i två steg. Den första frågan i mallen ber dig att skriva ner saker som
du önskar att dina kollegor visar hänsyn och förståelse för under projektet. Detta görs lämpligtvis
skriftligen innan mötet, så att du i lugn och ro kan fundera igenom det. Att alla i gruppen får en
bra bild av förutsättningarna är viktigt. Det är grunden till att komma fram till hur
arbetet kan utföras så att det fungerar så bra som möjligt för alla inblandade. I denna del, fundera
exempelvis på följande punkter:

\begin{itemize}
\item \textbf{Följande saker vill jag att min/mina kollegor visar hänsyn och förståelse för}

  Förbered! Skriv text här.

\item \textbf{Hur ska jag bete mig för att stötta min/mina kollegor utifrån sina förutsättningar?}

  Diskutera! Skriv text här.

\end{itemize}

\section{Hur vi arbetar tillsammans}

% När ni har fyllt i dokumentet kan instruktionerna nedan tas bort.

Under mötet, utgå ifrån hur ni har kommit fram till att ni bäst stöttar alla i gruppen i avsnitt
\ref{prereq}, och skriv ner hur ni bäst arbetar tillsammans för att kunna ge varandra det stöd ni
vill ge. Detta blir i slutändan gemensamma ''förhållningsregler'' för hur ni arbetar i gruppen, och
vad ni förväntar er av varandra.


\begin{itemize}
\item \textbf{Vilka tider arbetar vi, och vilka tider är vi nåbara utöver detta?}

  Skriv text här.

\item \textbf{Hur kommunicerar vi med varandra? Vilka verktyg/kanaler använder vi? Hur och när är det okej att vi avbryter varandra?}

  Skriv text här.

\item \textbf{Hur gör vi för att ge varandra möjlighet att framföra åsikter och tankar om uppgifter och idéer till arbetet?}

  Skriv text här.

\item \textbf{Hur ofta tar vi paus? Ska vi hjälpas åt att påminna varandra om att ta paus?}

  Skriv text här.

\item \textbf{Arbetar vi tillsammans med uppgifter, eller var för sig?}

  Skriv text här.

\item \textbf{Hur bestämmer vi vem som gör vad?}

  Skriv text här.

\item \textbf{Hur specifierar vi vad som ingår i varje uppgift, och när den är klar?}

  Skriv text här.

\item \textbf{Hur snabbt förväntar vi oss att en uppgift kan vara klar?}

  Skriv text här.

\item \textbf{Hur håller vi reda på att uppgifter vi identifierat inte glöms bort?}

  Skriv text här.

\end{itemize}

\section{Om jag tycker att något inte fungerar}

% När ni har fyllt i dokumentet kan instruktionerna nedan tas bort.

Fundera till sist på hur ni i gruppen vill hantera saker som inte fungerar. Tänk på att
kursledningen gärna hjälper till med tips och råd vid hantering av problem. Ni får gärna får ha
med det som en del av era lösningar. Vi ser dock gärna att ni försöker hantera problem själva först.

\begin{itemize}
\item \textbf{Vad gör vi om någon kommer sent?}

  Skriv text här.

\item \textbf{Vad gör vi om någon inte slutför sina uppgifter?}

  Skriv text här.

\item \textbf{Vad gör vi om arbetsfördelningen blir ojämn?}

  Skriv text här.

\item \textbf{Hur tar vi upp ett problem med berörda personer?}

  Skriv text här.

\item \textbf{Hur ger jag kritik och beröm till andra personer i gruppen?}

  Skriv text här.

\end{itemize}

\section{Utvärdering}

% När ni har fyllt i dokumentet kan instruktionerna nedan tas bort.

Vid utsatt tid: utvärdera hur gruppkontraktet har följts, fundera på ifall något i kontraktet
behöver ändras, eller om något nytt behöver läggas till.

\begin{itemize}
\item \textbf{När ska vi påminna oss om gruppkontraktet och utvärdera hur det fungerat?}

  Skriv text här.

\end{itemize}

\end{document}
