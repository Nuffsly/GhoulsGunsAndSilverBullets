\documentclass{mall}

\newcommand{\version}{Version 1.0}
\author{Martin Kuiper, \url{marku849@student.liu.se}\\
  Jim Teräväinen, \url{jimte145@student.liu.se}}
\title{Gruppkontrakt}
\date{\today}
\rhead{}


\begin{document}
\projectpage

\section{Förutsättningar}
\label{prereq}

\begin{itemize}
\item \textbf{Följande saker vill jag att min/mina kollegor visar hänsyn och förståelse för}

  - Jag vill att fredagar är utan matlåda.

  - Jag vill ha flexibla arbetstider men att vi siktar på 8 timmar per dag (föreläsningar inkluderade).

  - Jag vill att allt arbete är över minimumkrav.

  - Jag vill ha minst 45 minuter lunchrast utan tal om arbete per dag.

  - Jag vill förstå hur saker fungerar och ställa frågor tills jag gör det.

  - Jag vill att arbete som görs enskilt förklaras för hela gruppen tills alla förstår.

  - Jag vill ha öppen och direkt kommunikation kring kritik och diskutera lösningar.

  - Jag vill att projeketet inte blir för stort och förblir genomförbart på den givna tiden.

\item \textbf{Hur ska jag bete mig för att stötta min/mina kollegor utifrån sina förutsättningar?}

  Vi kommer att respektera om den ena parten vill äta lunch ute på fredagar och att rasten därmed kan bli lite längre. 

Vi kommer att försöka starta dagarna mellan 8 och 9 och från den tiden spendera 8 timmar. Vid andra tider ska detta kommuniceras i tid.

Vi kommer båda att sikta på att uppnå en nivå som uppnår högre betyg i alla delar av projektet. Vid behov och diskussion så kan detta mål sänkas eller ignoreras för att uppnå godkänd nivå.

Vi kommer aktivt att inkludera alla gruppmedlemmar i koden vi skriver och metoderna vi använder så att alla kan förstå och ta till sig.

Vi kommer att ge varandra konstruktiv kritik och respektera varandras känslor, samt visa hänsyn till att ett problem kan ha fler lösningar.

Vi ser till att välja en design på projektet som vi kan uppnå under den givna tiden.

\end{itemize}

\section{Hur vi arbetar tillsammans}

% När ni har fyllt i dokumentet kan instruktionerna nedan tas bort.

Under mötet, utgå ifrån hur ni har kommit fram till att ni bäst stöttar alla i gruppen i avsnitt
\ref{prereq}, och skriv ner hur ni bäst arbetar tillsammans för att kunna ge varandra det stöd ni
vill ge. Detta blir i slutändan gemensamma ''förhållningsregler'' för hur ni arbetar i gruppen, och
vad ni förväntar er av varandra.


\begin{itemize}
\item \textbf{Vilka tider arbetar vi, och vilka tider är vi nåbara utöver detta?}

  Från att vi börjar försöker vi arbeta 8 timmar per dag. Start tid bör vara mellan 8-9 varje dag om inget annat kommuniceras i förväg.

\item \textbf{Hur kommunicerar vi med varandra? Vilka verktyg/kanaler använder vi? Hur och när är det okej att vi avbryter varandra?}

  Vi kommunicerar i person eller över direktmeddelande på Discord. Vid behov kan röstchatt eller telefonsamtal användas. Dessa bör nyttjas vid rimliga tider t.ex inte efter 22:00 en vardag eller före 07:00. Mellan dessa tider bör inga svar förväntas om meddelande skickas.

\item \textbf{Hur gör vi för att ge varandra möjlighet att framföra åsikter och tankar om uppgifter och idéer till arbetet?}

  Vi strävar aktivt efter att fråga varandra om vi har andra åsikter och lyssnar när den andra parten framför sina åsikter.

\item \textbf{Hur ofta tar vi paus? Ska vi hjälpas åt att påminna varandra om att ta paus?}

  Vi tar paus vid behov och för lunch. Behov av paus respekteras av båda parter då vi inser värdet av tillfälliga avbrott.

\item \textbf{Arbetar vi tillsammans med uppgifter, eller var för sig?}

  I största mån arbetar vi tillsammans men vid speciella tillfällen och där det kan anses berättigat kan arbete ske enskilt.

\item \textbf{Hur bestämmer vi vem som gör vad?}

  Vi diskuterar uppgifter och försöker dela upp dem rättvist enligt vad som är svårt, lätt, tråkigt, roligt, tidskrävande och intressant för respektive part.

\item \textbf{Hur specifierar vi vad som ingår i varje uppgift, och när den är klar?}

  Varje uppgift bör delas upp i rimliga delmål innan arbetet startar. Uppgifterna bör baseras på innehållet i kravspecifikationen.

\item \textbf{Hur snabbt förväntar vi oss att en uppgift kan vara klar?}

  Uppgifter bör ha uppskattad tidsåtgång som ska försöka följas. När tiden som uppskattats inte räcker till kommuniceras detta och omplanering utförs.

\item \textbf{Hur håller vi reda på att uppgifter vi identifierat inte glöms bort?}

  Vi utnyttjar Gitlabs arbetstavle funktion. Varje delmål bör registreras som ett 'issue' på Gitlab och flyttas inom tavlan enligt vilket stadie den befinner sig i.

\end{itemize}

\section{Om jag tycker att något inte fungerar}

% När ni har fyllt i dokumentet kan instruktionerna nedan tas bort.

Fundera till sist på hur ni i gruppen vill hantera saker som inte fungerar. Tänk på att
kursledningen gärna hjälper till med tips och råd vid hantering av problem. Ni får gärna får ha
med det som en del av era lösningar. Vi ser dock gärna att ni försöker hantera problem själva först.

\begin{itemize}
\item \textbf{Vad gör vi om någon kommer sent?}

  Skriv text här.

\item \textbf{Vad gör vi om någon inte slutför sina uppgifter?}

  Skriv text här.

\item \textbf{Vad gör vi om arbetsfördelningen blir ojämn?}

  Skriv text här.

\item \textbf{Hur tar vi upp ett problem med berörda personer?}

  Skriv text här.

\item \textbf{Hur ger jag kritik och beröm till andra personer i gruppen?}

  Skriv text här.

\end{itemize}

\section{Utvärdering}

% När ni har fyllt i dokumentet kan instruktionerna nedan tas bort.

Vid utsatt tid: utvärdera hur gruppkontraktet har följts, fundera på ifall något i kontraktet
behöver ändras, eller om något nytt behöver läggas till.

\begin{itemize}
\item \textbf{När ska vi påminna oss om gruppkontraktet och utvärdera hur det fungerat?}

  Skriv text här.

\end{itemize}

\end{document}
