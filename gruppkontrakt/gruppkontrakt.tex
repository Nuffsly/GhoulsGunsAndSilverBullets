\documentclass{mall}

\newcommand{\version}{Version 1.0}
\author{Martin Kuiper, \url{marku849@student.liu.se}\\
  Jim Teräväinen, \url{jimte145@student.liu.se}}
\title{Gruppkontrakt}
\date{\today}
\rhead{}


\begin{document}
\projectpage

\section{Förutsättningar}
\label{prereq}

\begin{itemize}
\item \textbf{Följande saker vill jag att min/mina kollegor visar hänsyn och förståelse för}

  - Jag vill att fredagar är utan matlåda.

  - Jag vill ha flexibla arbetstider men att vi siktar på 8 timmar per dag (föreläsningar inkluderade).

  - Jag vill att allt arbete är över minimumkrav.

  - Jag vill ha minst 45 minuter lunchrast utan tal om arbete per dag.

  - Jag vill förstå hur saker fungerar och ställa frågor tills jag gör det.

  - Jag vill att arbete som görs enskilt förklaras för hela gruppen tills alla förstår.

  - Jag vill ha öppen och direkt kommunikation kring kritik och diskutera lösningar.

  - Jag vill att projeketet inte blir för stort och förblir genomförbart på den givna tiden.

\item \textbf{Hur ska jag bete mig för att stötta min/mina kollegor utifrån sina förutsättningar?}

  Vi kommer att respektera om den ena parten vill äta lunch ute på fredagar och att rasten därmed kan bli lite längre. 

Vi kommer att försöka starta dagarna mellan 8 och 9 och från den tiden spendera 8 timmar. Vid andra tider ska detta kommuniceras i tid.

Vi kommer båda att sikta på att uppnå en nivå som uppnår högre betyg i alla delar av projektet. Vid behov och diskussion så kan detta mål sänkas eller ignoreras för att uppnå godkänd nivå.

Vi kommer aktivt att inkludera alla gruppmedlemmar i koden vi skriver och metoderna vi använder så att alla kan förstå och ta till sig.

Vi kommer att ge varandra konstruktiv kritik och respektera varandras känslor, samt visa hänsyn till att ett problem kan ha fler lösningar.

Vi ser till att välja en design på projektet som vi kan uppnå under den givna tiden.

\end{itemize}

\section{Hur vi arbetar tillsammans}

\begin{itemize}
\item \textbf{Vilka tider arbetar vi, och vilka tider är vi nåbara utöver detta?}

  Vi har som mål att börja arbeta mellan klockan 8 och 9 varje vardag och hålla på i ca 8 timmar. Undantag till detta kommuniceras i förväg. 

\item \textbf{Hur kommunicerar vi med varandra? Vilka verktyg/kanaler använder vi? Hur och när är det okej att vi avbryter varandra?}

  Vi kommunicerar i person eller över direktmeddelande på Discord. Vid behov kan röstchatt eller telefonsamtal användas. Dessa bör nyttjas vid rimliga tider, dvs inte mellan 22:00 och 07:00. Mellan dessa tider bör heller inga svar förväntas om meddelande skickas.

\item \textbf{Hur gör vi för att ge varandra möjlighet att framföra åsikter och tankar om uppgifter och idéer till arbetet?}

  Vi strävar aktivt efter att fråga varandra om vi har andra åsikter och lyssnar när den andra parten framför sina åsikter.

\item \textbf{Hur ofta tar vi paus? Ska vi hjälpas åt att påminna varandra om att ta paus?}

  Vi tar paus vid behov och för lunch. Behov av paus respekteras av båda parter då vi inser värdet av tillfälliga avbrott.

\item \textbf{Arbetar vi tillsammans med uppgifter, eller var för sig?}

  I största mån arbetar vi tillsammans men vid speciella tillfällen och där det kan anses berättigat kan arbete ske enskilt.

\item \textbf{Hur bestämmer vi vem som gör vad?}

  Vi diskuterar uppgifter och försöker dela upp dem rättvist enligt vad som är svårt, lätt, tråkigt, roligt, tidskrävande och intressant för respektive part.

\item \textbf{Hur specifierar vi vad som ingår i varje uppgift, och när den är klar?}

  Varje uppgift bör delas upp i rimliga delmål innan arbetet startar. Uppgifterna bör baseras på innehållet i kravspecifikationen.

\item \textbf{Hur snabbt förväntar vi oss att en uppgift kan vara klar?}

  Uppgifter bör ha uppskattad tidsåtgång som ska försöka följas. När tiden som uppskattats inte räcker till kommuniceras detta och omplanering utförs.

\item \textbf{Hur håller vi reda på att uppgifter vi identifierat inte glöms bort?}

  Vi utnyttjar Gitlabs arbetstavla-funktion. Varje delmål bör registreras som ett 'issue' på Gitlab och flyttas inom tavlan enligt vilket stadie den befinner sig i.

\end{itemize}

\section{Om jag tycker att något inte fungerar}

\begin{itemize}
\item \textbf{Vad gör vi om någon kommer sent?}

  Kommer en gruppmedlem så sent och med sån frekvens att det blir ett problem så får personen en varning. Fortgår beteendet kommer kursledningen kontaktas.

\item \textbf{Vad gör vi om någon inte slutför sina uppgifter?}

  Om uppgifter inte utförs inom rimlig tid samtalar gruppen om det och försöker hitta lösningar. Hjälper detta inte kontaktas kursledningen.

\item \textbf{Vad gör vi om arbetsfördelningen blir ojämn?}

  Om arbetsfördelningen upplevs ojämn av någon i gruppen diskuterar vi det och undersöker om uppgifter behöver omfördelas. Vi planerar också att arbeta tillsammans eller samtidigt en stor del av tiden, vilket bör verka för en jämnare arbetsfördelning. 

\item \textbf{Hur tar vi upp ett problem med berörda personer?}

  Vi meddelar personen att vi upplever ett problem med samarbetet. Den andra parten åtar sig att lyssna och försöka lösa problemet. 

\item \textbf{Hur ger jag kritik och beröm till andra personer i gruppen?}

  Kritik ges med omtanke och ödmjukhet. Vi anser att samtal om de bästa sätten att lösa olika problem är ett värdefullt sätt att lära sig. 
Vi har som mål att ge beröm ofta.

\end{itemize}

\newpage

\section{Utvärdering}

\begin{itemize}
\item \textbf{När ska vi påminna oss om gruppkontraktet och utvärdera hur det fungerat?}

  Vi planerar att titta igenom, utvärdera och eventuellt revidera gruppkontraktet den 1a December 2020. Vid det tillfället får samtliga medlemmar framföra sina tankar om hur det fungerat och potentiellt ge förslag på ändringar.

\end{itemize}

\end{document}

